
\addcontentsline{toc}{chapter}{Abstract}
\chapter*{\label{intro}Abstract}

\setcounter{equation}{0}
\setcounter{table}{0}
\setcounter{figure}{0}
%\baselineskip 24pt

Incidence of anastomotic leak (AL) in patients requiring anastomosis is around 10\%. Patients who have AL have a statistically significant increase in mortality rate. The main confirmation of AL is through elevated CRP along with patient reported symptoms such as pain and fever. CRP measurement is a systemic marker and patients with AL can be identified two days post operative. Current literature indicates that measuring abundance of unconventional T cells found in the peritoneal fluid can predict complications in peritonitis patients; we theorise that we can apply similar methods to an AL clinical cohort. Finding markers with earlier detection and higher specificity may improve patient outcomes and reduce AL associated mortality rate.

\begin{enumerate}
    \item Compare gene expression changes of $\gamma\delta$ T cells which enable them to polarise CD4$^+$ T cells in response to bacterial infection.
    \item Assess if immunophenotyping data, specifically the abundance of unconventional T cells, correlates with various clinical endpoints. This evaluation is a preliminary step towards identifying an immune signature with high specificity for determining the risk of developing AL.
    \item Propose an AL immune signature to detect risks earlier and with higher specificity than markers used in current practice.
\end{enumerate}

\clearpage

\addcontentsline{toc}{chapter}{Glossary}
\chapter*{\label{intro}Glossary}

\setcounter{equation}{0}
\setcounter{table}{0}
\setcounter{figure}{0}
%\baselineskip 24pt

\begin{description}
    \item[CRC] Colorectal cancer
    \item[AL] Anastomotic leak
    \item[CRP] C-reactive protein
    \item[HMB-PP] (E)-4-hydroxy-3-methyl-but-2-enyl pyrophosphate
    \newline
    \begin{figure}[h]
        \begin{center}
        \setchemfig{chemfig style={line width=1pt}}
        \scalebox{0.7}{\chemfig{HO-[::-30]-[::+60](-[::+60])=[::-60]-[::+60]-[::-60]O-[::+45]P(=[::-15]O)(-[::-60]OH)-[::90]O-[::-60]P(-[::15]OH)(-[::75]HO)=O}}
        \end{center}
      \caption{Chemical structure of HMB-PP}
      \label{hmbpp_structure}
    \end{figure}
    \item[APC] Antigen presenting cell
    \item[TCR] T cell receptor
    \item[MAIT] Mucosal-associated invariant T cell
    \item[NKT] Natural killer T cell
    \item[BTN] Butyrophilin
    \item[PD] Peritoneal dialysis
    \item[ILC] Innate lymphoid cells
\end{description}

 
